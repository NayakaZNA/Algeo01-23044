\section{Deskripsi Masalah}
\subsection{Gambaran Umum}
Sistem persamaan linear (SPL) adalah himpunan persamaan-persamaan linear yang menggunakan peubah yang sama. SPL kerap ditemui di beragam bidang keilmuan untuk beragam keperluan. Atas dasar itu, diperlukanlah suatu metode untuk menyelesaikan SPL. 

Dalam mata kuliah IF$2123$ Aljabar Linear dan Geometri, metode-metode yang telah dipelajari adalah (1) metode eliminasi Gauss, (2) metode eliminasi Gauss-Jordan, (3) metode matriks balikan, dan (4) metode Cramer. Diberikan sebuah SPL, terdapat tiga kemungkinan yang kita miliki: SPL mempunyai solusi unik/tunggal, tak berhingga, atau bahkan tidak ada sama sekali.

\subsection{Persoalan}
\par Pada tugas besar ini, kita akan mengimplementasikan keempat\footnote{Secara implisit, ini berarti kita juga mengimplementasikan fungsi untuk menghitung determinan dengan metode reduksi baris maupun ekspansi kofaktor, \textit{adjoin}, dan balikan dari suatu matriks.} metode tersebut dalam sebuah pustaka (\textit{library}) Java untuk kemudian diterapkan dalam beberapa persoalan, mencakup:
\begin{enumerate}
    \item Menyelesaikan persoalan dalam bentuk SPL
    
    Diberikan masukan berupa SPL dalam bentuk matriks \textit{augmented}, program harus mampu menentukan solusi dari SPL yang diberikan.
    
    \item Menghitung determinan dan balikan dari suatu matriks
    \item Interpolasi polinom dan \textit{bicubic spline}
    
    Diberikan $n+1$ buah titik berbeda, $(x_0, y_0)$, $(x_1, y_1)$, $\ldots$, $(x_n, y_n)$, kita dapat menentukan polinom $p_n(x)$ yang melewati semua titik-titik tersebut sedemikian rupa sehingga $y_i = p_n(x_i)$ untuk $0 \leq i \leq n$.

    \item Regresi berganda (linear dan kuadratik)
    
    Regresi adalah metode untuk menentukan \textit{best-fit line} dari sekumpulan titik yang diberikan. Bedanya dengan interpolasi adalah \textit{best-fit line} ini tidak harus melalui seluruh titik yang diberikan.
\end{enumerate}

    Sebagai tambahan, dalam tugas besar ini kami juga membuat tampilan antarmuka (\textit{graphical interface}, GUI) dan video penjelasan. Terakhir, sebagai aplikasi dari interpolasi \textit{bicubic spline}, kami juga akan mengimplementasikan \textit{image resizing and stretching}.

\pagebreak
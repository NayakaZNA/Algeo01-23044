\section{Penutup}
\subsection{Kesimpulan}
Secara keseluruhan, kami berhasil menyelesaikan seluruh komponen nonbonus dengan baik. Untuk tugas bonus, kami tidak mengerjakan GUI karena dirasa cukup sukar dan penggunaan \textit{command line interface} dirasa sudah cukup memfasilitasi penggunaan program ini. Kami juga berusaha untuk terus mengoptimasi kode-kode yang sudah kami buat sebelumnya supaya program dapat bekerja secara efisien.

Beberapa kesimpulan yang didapatkan selama pengerjaan, baik secara teoretis maupun praktis, adalah bahwa eliminasi Gauss ternyata lebih efisien dibandingkan kaidah Cramer. Dalam menghitung balikan matriks pun ternyata metode Gauss-Jordan juga lebih optimal dibandingkan cara adjoin.

\subsection{Saran}
Hemat kami, untuk tugas besar selanjutnya ada baiknya dicantumkan banyak referensi yang bisa digunakan untuk mengembangkan wawasan dan menyelesaikan persoalan terkait tugas besar tersebut. Referensi yang diberikan bisa menjadi awalan bagi kami untuk bereksplorasi dan mencari referensi lebih banyak lagi. Selain itu, nilai untuk bonus GUI terlalu kecil sehingga sebagian peserta kelas mungkin merasa kurang \textit{worth it} untuk mengerjakannya.

\subsection{Komentar}
    Tugas besar ini tidak hanya memperkaya khazanah keinformatikaan kami, tetapi juga mempererat persahabatan di antara kami bertiga.

\begin{rmr}{Muhammad Luqman Hakim}{}
    Tugas besar yang mantap.
\end{rmr}

\begin{rmr}{Zulfaqqar Nayaka Athadiansyah}{}
    Selagi menelusuri referensi untuk tugas besar ini, saya menemukan banyak buku-buku yang menarik.
\end{rmr}

\begin{rmr}{Farrel Athalla Putra}{}
    Saya senang mengerjakan tugas besar ini. Akan tetapi, menurut saya untuk regresi kuadratik berganda dan interpolasi \textit{bicubic spline} perlu diberi referensi lebih banyak karena mencari pembahasan mendalam tentang topik ini relatif susah. \footnote{Terima kasih sudah mengerjakan bagian itu, Luqman!}
\end{rmr}

\subsection{Refleksi}
Kami berhasil membagi tugas dengan bijak dan sigap dalam mengerjakan bagian masing-masing. Laporan ini sudah dikerjakan dengan progres lebih dari $50\%$ hanya beberapa hari setelah rilis tugas besar. \textit{Class} \verb+MatrixADT+, balikan, SPL, dan determinan juga dikerjakan dengan cepat. Akan tetapi, masih ada sejumlah \textit{blunder} yang kami lakukan, yakni menunda-nunda pekerjaan selama berhari-hari dan bersantai sejenak, tidak merencanakan struktur folder program dari awal, serta beberapa kesalahan minor lainnya. Masih ada penggunaan maupun pembuatan fungsi yang redundan. Terlepas dari itu, kami bersyukur bisa menyelesaikan tugas besar ini.

\pagebreak